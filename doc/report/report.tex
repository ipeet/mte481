\documentclass[oneside,final]{report}

\usepackage{graphicx}
\usepackage[parfill]{parskip}

\begin{document}
 

% ================ Front Matter ================
% Suppress page number for title page and letter of submittal
\pagestyle{empty}

% ==== Title Page
\begin{flushright}
 \begin{LARGE}
  \textbf{UNIVERSITY OF WATERLOO}
 \end{LARGE}

 \begin{large}
  Faculaty of Mechatronics Engineering\\[4cm]
 \end{large}

 \begin{LARGE}
  \textbf{
    Fourth Year Design Project \\[0.0cm]
    Final Design Report
  }
 \end{LARGE}

 \vfill

  Prepared By: \\[0.2cm]
  Iain Peet - 20252201\\
  Jordan Valentin - xxxxxxxx\\
  Rowan Head-Marsden - xxxxxxxx\\
  \today
\end{flushright}
\clearpage

% ==== Letter of Transmittal
\today \\[0.5cm]

Faculty of Mechatronics Engineering \\
University of Waterloo \\
Waterloo, Ontario \\

To Whom It May Concern:

Lorem ipsum dolor est... \\[0.5cm]

Sincerely, \\[1cm]

Iain Peet \hspace{4cm} Jordan Valentin \hspace{4cm} Rowan Head-Marsden
\clearpage

% Resume page numbering
\pagestyle{plain}
\setcounter{page}{1}
\pagenumbering{roman}

% ==== Table of Contents
\tableofcontents

% ==== List of Figures
\listoffigures
\addcontentsline{toc}{chapter}{List of Tables and Figures}

\chapter*{Executive Summary}
\addcontentsline{toc}{chapter}{Summary}

% ====================== Report Body =================================
\clearpage
\setcounter{page}{1}
\pagenumbering{arabic}
\pagestyle{headings}

\chapter{Introduction}

\section{Background}
Powered wheelchairs have obvious benefits for the mobility of the physically disabled.  As relatively heavy, powered vehicles, operation of these wheelchairs also carries obvious risk.  Thus, operators must be capable of safe operations.

Irene Ruel, a physiotherapist in the employ of the British Columbia Interior Health Authority, who has previously been responsible for powered wheelchair assessments in an extended care facilities, was consulted regarding this problem.  By her account, the risk that powered wheelchair operators will collide with other patients is a serious concern, which frequently results in the denial of requests for powered wheelchairs.

\section{Need Statement}
Access to powered wheelchairs may be restricted due to concerns about a patient's capability for safe operation.

\section{Objectives and Constraints}
The objectives of this design problem are as follows:

\begin{itemize}
 \item Improve the safety of powered wheelchairs, for the occupant, and for nearby pedestrians.
 \item Make powered wheelchairs accessible to people who would otherwise be denied due to safety concerns. 
 \item Assist wheelchair users with difficult tasks, such as precise positioning and movement in constrained spaces.  
\end{itemize}

The constraints of the design are as follows:

\begin{itemize}
 \item A specialized, controlled operating environment must not be required. 
 \item It must be possible to integrate with existing wheelchairs. 
 \item Cost must be kept below \$500.
\end{itemize}


\section{Patents}
% TODO(jordan)


\chapter{Proposed Solution}

\section{Potential Solutions}

\section{Solution Selection}


\chapter{Detailed Design}

\section{Summary}

\section{Mechanical Design}

\section{Systems Integration}

\section{Software Design}


\chapter{Schedule and Budget}

\section{Prototype Schedule}

\section{Bill of Materials}


\chapter{Conclusions and Recommendations}

\chapter*{Appendix A}
\addcontentsline{toc}{chapter}{Appendix A}

\begin{thebibliography}{9}
 \addcontentsline{toc}{chapter}{Bibliography}

 % NB: cite these with \cite{item}

 \bibitem{lv-ez4}
   \emph{LV-EZ4 Datasheet}, MaxBotix Inc, 2011

\end{thebibliography}


\end{document}